% Created 2020-09-25 Fri 14:56
% Intended LaTeX compiler: pdflatex
\documentclass[11pt]{article}
\usepackage[utf8]{inputenc}
\usepackage[T1]{fontenc}
\usepackage{graphicx}
\usepackage{grffile}
\usepackage{longtable}
\usepackage{wrapfig}
\usepackage{rotating}
\usepackage[normalem]{ulem}
\usepackage{amsmath}
\usepackage{textcomp}
\usepackage{amssymb}
\usepackage{capt-of}
\usepackage{hyperref}
\author{Max Fan}
\date{\today}
\title{Housing Allocation Optimization}
\hypersetup{
 pdfauthor={Max Fan},
 pdftitle={Housing Allocation Optimization},
 pdfkeywords={},
 pdfsubject={},
 pdfcreator={Emacs 26.3 (Org mode 9.3.7)}, 
 pdflang={English}}
\begin{document}

\maketitle

\section{Motivations}
\label{sec:org47c1139}
Traditionally, housing allocation at Choate Rosemary Hall was done by hand.
Decades ago, this process was done through a lottery, with students randomly selecting numbers out of a hat.
The students would be allocated to the house of their choosing in the ordering of the numbers they received.
This system was later amended to allow for more flexibility, giving Choate students the system, they have today -- a "random" lottery that takes into account everyone's preferences and priorities.

The optimization techniques available have drastically improved since the 20th century.
It is now feasible for normal laptops to churn through hundreds of thousands of possibilities per second.
In addition to the computational leaps made within the past century, there has been great development in the theory of optimization and resource allocation.
We aim to examine and develop several optimization techniques that can take advantage of these recent developments.

\section{Introduction}
\label{sec:orgc44ba09}
It is important to understand that the purpose of our project is not to completely replace the current system on hand. We wish to create a tool that the Deans can use to assist them in their task to create the housing allocation. Our goal is to create an algorithm that can create a housing allocation within seconds and have it satisfy the majority of the student’s preferences. The finishing touches are still expected to be finished with human hands.


\subsection{The Deans' Algorithm}
\label{sec:orgea12a01}
We define the Dean’s Algorithm to be the current system that Choate uses for the housing allocations. During the span of our project, our goal was to improve on the current system and provide aid to it. To do that, it is important to understand the current system.
The Dean’s Algorithm starts by asking all students to rank their top four preferred houses. The students are then randomly distributed through the available houses using a computer system, then the allocation is finished by hand based on how the student’s ranked their preferences. Of course, there are many other nuances that change the results.

The Lottery system
There is a lottery system put in place to help your chances of obtaining the house you want. Each student chooses a random number, and those who choose a larger number are given priority over those who chose a worse number. When students are on leave (medical or otherwise) for a year, students are unable to participate in the lottery. The school identifies what would be a good fit for the student based on what is known about them. 
The purpose of the lottery system is to not only make it fun for the students to bet their luck, but it is also to create an element of randomness. The school wishes for diversity within each house in hopes that different groups of people can gather together and become friends.

Nuances
Fourth formers are normally given priority over 3rd formers and new students. In addition, fourth formers are in a tiered lottery that can be separated into three different tiers, as indicated below (refers to roommate status).
- Tier 1 status. Both students choose to stay in the same house and have priority for beds
- Tier 2 status. One student chooses the stay in the same house and their roommate is coming from a different house
- Tier 3 status. Neither student are returning members of the house

The school does not allow students to fill out all their living choices as singles. A student is only allowed to fill out one or two houses as singles if they would like to at all.

\subsection{MCMC}
\label{sec:orga95e088}

In this project, we used Markov chain Monte Carlo methods to distribute students into their respective housing. MCMC is a generic method for approximate sampling from an arbitrary distribution. The main idea is to generate a Markov chain whose limiting distribution is equal to the described distribution.
Metropolis-Hastings Algorithm is a specific type of MCMC method that we used in this project.
Metropolis-Hastings Algorithm:
The algorithm of Metropolis-Hastings MCMC is as the following: (P(a) is the probability of a)
Initialize with some random state X0
From the current state, generate a new state X’
To decide whether X’ is accepted as the next state: if P(X1) <= P(X’), it is accepted 100%. If P(X1) > P(X’), it is accepted with a probability of P(X’)/P(X1) % and rejected otherwise.
Repeat the process of 2 and 3 as many times as needed.


\subsection{Constraint Solvers}
\label{sec:org640dd4f}

\section{Methodology}
\label{sec:orgc0ba115}
To test our algorithms, we first needed sample inputs. We did this by generating multiple yaml files using generating_ballots.py, which randomly generated ballots - the randomness was created using Zipf’s law, a reliable statistical model that creates probability distributions. We also made each house have different popularity - there always were very popular houses, and not so popular houses. 
The input yaml file is read by input.rs. 

data_output.rs creates a yaml file that contains the information for the final housing allocation.


\section{Conclusion}
\label{sec:orgb6da137}
\end{document}
